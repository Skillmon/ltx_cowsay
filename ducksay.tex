\documentclass[]{article}

\usepackage{ducksay}
\newcommand*{\availableAnimal}[1]{\par\makebox[0pt][l]{\ducksay[#1]{#1}}\mbox{}\\[1ex]}
\newcommand*{\anml}{\texttt{<animal>}}
\newcommand*{\msg}{\texttt{<message>}}

\begin{document}
\begin{titlepage}%>>>
  \makeatletter
  \centering
  %\mbox{}\vfill
  \Large
    \ducksay[duck,bubble=\huge]{This is ducksay!}\\
  \vfill
  \normalsize
  \hspace*{-2cm}
    \ducksay[cow,bubble=\large]{\ducksay@version}\\
  \small
  \vspace*{-5cm}\hspace*{5cm}
    \ducksay[small-duck,bubble=\normalsize]{But which Version?}
  \mbox{}\hfil
  \vspace{2cm}
  \vfill
  \vfill
  \hspace*{-0cm}
  \large
  \smash{%
    \ducksay[r2d2,bubble=\large]{by Jonathan P. Spratte}}
  \small
    \ducksay[hedgehog,bubble=\normalsize]{Today is \ducksay@date}
  \makeatother
\end{titlepage}%<<<
\section{Macros}
\bgroup
\parindent=-2em
\parskip=1em
\hspace*{-2em}%
\verb|\ducksay[<options>]{<message>}|\\
  options might include any of the options described in
  section~\ref{sec:options}. Prints an \anml\ saying \msg. \msg\ is not
  verbatim.

\verb|\duckthink[<options>]{<message>}|\\
  not implemented

\verb|\DefaultAnimal{<animal>}|\\
  use the \anml\ if no optional argument is given to \verb|\ducksay|. Package
  default is duck. You might specify more options than \anml\ with this, but it
  must contain an \anml, otherwise you have to specify an \anml\ in each call of
  \verb|\ducksay|.

\verb|\AddAnimal(*){<animal>}<ascii-art>|\\
  adds \anml\ to the known animals. \texttt{<ascii-art>} is multi-line verbatim
  and therefore should be delimited either by matching braces or by anything
  that works for \verb|\verb|. If the star is given \anml\ is the new default.
  One space is added to the begin of \anml. For example, snowman is added with:
\begin{verbatim}
\AddAnimal{snowman}
{  \
    \_[_]_
      (")
   >-( : )-<
    (__:__)}
\end{verbatim}
\egroup

\section{Options}\label{sec:options}
The following Options are available to \verb|\ducksay| and \verb|\duckthink|:

\bgroup
\parindent=-2em
\parskip=1em
\anml\\
  One of the animals listed in section~\ref{sec:animals} or any of the ones
  added with \verb|\AddAnimal|.

\texttt{bubble=\#1}\\
  use \texttt{\#1} in a group with the bubble (for font switches).
\egroup
  
\section{Defects}
\begin{itemize}
  \item no line wrapping
  \item no \verb|\duckthink|
\end{itemize}

\twocolumn
\section{Available Animals}\label{sec:animals}
\small
\availableAnimal{duck}
\availableAnimal{small-duck}
\availableAnimal{cow}
\availableAnimal{tux}
\availableAnimal{pig}
\availableAnimal{frog}
\availableAnimal{snowman}
\availableAnimal{r2d2}
\availableAnimal{head-in}
\availableAnimal{hedgehog}
\availableAnimal{kangaroo}
\availableAnimal{vader}
\onecolumn
\availableAnimal{yoda}

\clearpage
\thispagestyle{empty}
\bgroup
\Huge
\mbox{}\vfill
\centering
\ducksay{Have fun with it!}
\vfill
\egroup
\end{document}

% vim: fdm=marker foldmarker=>>>,<<<
