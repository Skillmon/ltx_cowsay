% \iffalse meta-comment
%
% File: ducksay.dtx Copyright (C) 2017-2018 Jonathan P. Spratte
%
% This work  may be  distributed and/or  modified under  the conditions  of the
% LaTeX Project Public License (LPPL),  either version 1.3c  of this license or
% (at your option) any later version.  The latest version of this license is in
% the file:
%
%   http://www.latex-project.org/lppl.txt
%
% ------------------------------------------------------------------------------
%
%<*driver>^^A>>>
\def\nameofplainTeX{plain}
\ifx\fmtname\nameofplainTeX\else
  \expandafter\begingroup
\fi
\input l3docstrip.tex
\askforoverwritefalse
\preamble

--------------------------------------------------------------
ducksay -- cowsay for LaTeX
E-mail: jspratte@yahoo.de
Released under the LaTeX Project Public License v1.3c or later
See http://www.latex-project.org/lppl.txt
--------------------------------------------------------------

Copyright (C) 2017-2018 Jonathan P. Spratte

This  work may be  distributed and/or  modified under  the conditions  of the
LaTeX Project Public License (LPPL),  either version 1.3c  of this license or
(at your option) any later version.  The latest version of this license is in
the file:

  http://www.latex-project.org/lppl.txt

This work is "maintained" (as per LPPL maintenance status) by
  Jonathan P. Spratte.

This work consists of the file  ducksay.dtx
and the derived files           ducksay.pdf
                                ducksay.sty
                                ducksay.code.v1.tex
                                ducksay.code.v2.tex and
                                ducksay.animals.tex.

\endpreamble
% stop docstrip adding \endinput
\postamble
\endpostamble
\generate{\file{ducksay.sty}{\from{ducksay.dtx}{pkg}}}
\generate{\file{ducksay.code.v1.tex}{\from{ducksay.dtx}{code.v1}}}
\generate{\file{ducksay.code.v2.tex}{\from{ducksay.dtx}{code.v2}}}
\generate{\file{ducksay.animals.tex}{\from{ducksay.dtx}{animals}}}
\ifx\fmtname\nameofplainTeX
  \expandafter\endbatchfile
\else
  \expandafter\endgroup
\fi
%
\ProvidesFile{ducksay.dtx}
  [\csname ducksay@date\endcsname\ cowsay for LaTeX]
\documentclass{l3doc}
\usepackage[version=2]{ducksay}
\let\metaORIG\meta
\protected\def\meta #1{\texttt{\metaORIG{#1}}}
\DucksayOptions{arg=tab,msg-align=l,vpad=1}
\renewcommand*\thefootnote{\fnsymbol{footnote}}
\newcommand*\anml{\meta{animal}}
\newcommand*\msg{\meta{message}}
\newcommand*\PolesInfo
  {%
    \href{https://ctan.org/pkg/l3kernel}{\file{interface3.pdf}} and the
    documentation of \href{https://ctan.org/pkg/xcoffins}{\pkg{xcoffins}} for
    information about coffin poles.%
  }
\usepackage{enumitem}
\newenvironment{options}[1][]
  {%
    \begin{description}
      [
        style=nextline
        ,font=\normalfont\ttfamily
        ,labelindent=-.5\marginparwidth
        ,labelwidth=\dimexpr.5\marginparwidth-5pt\relax
        ,labelsep*=5pt
        ,leftmargin=!
        ,#1
      ]%
    \renewenvironment{options}[1][]
      {%
        \begin{description}
          [
            style=nextline
            ,font=\normalfont\ttfamily
            ,#1
          ]%
      }
      {\end{description}}%
  }
  {\end{description}}
\makeatletter
\newcommand*\availableAnimal[1]
  {%
    \@for\cs:=#1\do
      {%
        \ifx\cs\@empty\else
          \rlap{\expandafter\ducksay\expandafter[\cs]{\cs}}\hfill\null\par%
        \fi
      }%
  }
\def\@oddfoot
  {%
    \null\hfil
    \makebox[0pt][c]
      {%
        \smash
          {%
            \ducksay
              [dog,body=\tiny,wd=3,msg-align=c,out-v=T,MSG=\normalsize,vpad=0]
              {\thepage}%
          }%
      }%
    \hfil\null
  }
\let\@evenfoot\@oddfoot
\def\@oddhead
  {%
    \null\hfill\firstmark
  }
\let\@evenhead\@oddhead
\newcommand\DocImp{}
\newcommand\SetVersion[1]
  {%
    \clearpage
    \if\relax\detokenize{#1}\relax
      \markboth{}{}%
    \else
      \markboth{}{\DocImp\ of Version #1}%
    \fi
  }
\makeatother
\setcounter{secnumdepth}{5}
\setcounter{tocdepth}{5}
\NewDocumentCommand \tocmsg {}
  {%
    \marginpar
      {%
        \tiny
        \hfil
        \makebox[.\marginparwidth][l]
          {%
            \ducksay[head-in,MSG=\footnotesize,msg-align=c,align=t]
              {It's always\\good to\\keep the\\overview!}%
          }%
      }%
    \unskip
  }
\addtocontents{toc}{\protect\tocmsg\vspace*{-\baselineskip}}
\newcommand*\closingpage%>>>
  {%
    \clearpage
    \thispagestyle{empty}
    \bgroup
    \Huge
    \null\vfill
    \centering
    \makebox[0pt]{\duckthink{Who's gonna use it anyway?}}
    \vfill
    \hfill
    \smash
      {%
        \footnotesize
        \ducksay[small-yoda,wd=39,ht=3,msg-align=c,rel-align=r]
          {Hosted at\\\url{https://github.com/Skillmon/ltx_ducksay}\\it is.}%
      }
    \egroup
    \clearpage
  }%<<<
\begin{document}
  \DocInput{ducksay.dtx}
\end{document}
%</driver>^^A<<<
%<*pkg>^^A>>>
\NeedsTeXFormat{LaTeX2e}
\RequirePackage{xparse,l3keys2e}

\def\ducksay@version{2.0}
\def\ducksay@date{2018/09/21}

\ProvidesExplPackage
  {ducksay}           {\ducksay@date}
  {\ducksay@version}  {cowsay for LaTeX}

%</pkg>
%^^A<<<
% \fi
%
% \begin{titlepage}^^A>>>
%   \DucksayOptions{vpad=0}%
%   \makeatletter
%   \centering
%   \Large
%     \ducksay[duck,MSG=\huge,msg-align=c]{This is\\\pkg{ducksay}!}\\
%   \vfill
%   \normalsize
%   \hspace*{-2cm}
%     \ducksay[cow,MSG=\large,body-mirrored,body-align=r,msg-to-body=hc,out-h=r]
%       {v\ducksay@version}\\
%   \small
%   \vspace*{-5cm}\hspace*{5cm}
%     \ducksay[small-duck,MSG=\normalsize]{But which Version?}
%   \null\hfil
%   \vspace{2cm}
%   \vfill
%   \vfill
%   \hspace*{-0cm}
%   \large
%   \smash{%
%     \ducksay[r2d2,MSG=\large,body-mirrored,msg-to-body=hc,body-to-msg=r]
%       {by Jonathan P. Spratte}}
%   \small
%     \ducksay[hedgehog,MSG=\normalsize]{Today is \ducksay@date}
%   \makeatother
% \end{titlepage}^^A<<<
% \tableofcontents
%
% \begin{documentation}^^A>>>
%
% \section{Documentation}\def\DocImp{Documentation}%
%
% \subsection{Downward Compatibility Issues}
% \marginpar^^A>>>
%   {
%     \tiny
%     \ducksay[snail,MSG=\footnotesize,align=t]{Yep, I screwed up!}
%   }^^A<<<
%
% \begin{itemize}
%   \item Versions prior to v2.0 did use a regular expression for the option
%     |ligatures|, see \autoref{sec:options} for more on this issue. With v2.0
%     I do refer to the package's version, not the code variant which can be
%     selected with the |version| option.
%   \item In a document created with package versions prior to v2.0 you'll have
%     to specify the option |version=1| in newer versions to make those old
%     documents behave like they used to. 
% \end{itemize}
%
% \subsection{Shared between versions}
%
% \subsubsection{Macros}^^A>>>
% \marginpar^^A>>>
%   {
%     \tiny
%     \hfill
%     \ducksay[bunny,MSG=\footnotesize,align=t,body-mirrored]
%       {Macros for everyone!}
%   }^^A<<<
%
% A careful reader might notice that in the below list of macros there is no
% \cs{ducksay} and no \cs{duckthink} contained. This is due to differences
% between the two usable code variants (see the |version| key in
% \autoref{sec:options} for the code variants, \autoref{sec:macros1} and
% \autoref{sec:macros2} for descriptions of the two macros).
%
% \begin{function}{\DefaultAnimal}^^A>>>
%   \begin{syntax}
%     \cs{DefaultAnimal}\marg{animal}
%   \end{syntax}
%   use the \anml\ if none is given in the optional argument to \cs{ducksay}
%   or \cs{duckthink}. Package default is |duck|.
% \end{function}^^A<<<
%
% \begin{function}{\DucksayOptions}^^A>>>
%   \begin{syntax}
%     \cs{DucksayOptions}\marg{options}
%   \end{syntax}
%   set the defaults to the keys described in \autoref{sec:options},
%   \autoref{sec:options1} and \autoref{sec:options2}. Don't use an \anml\ here,
%   it has no effect.
% \end{function}^^A<<<
%
% \begin{function}{\AddAnimal}^^A>>>
%   \begin{syntax}
%     \cs{AddAnimal}\meta{*}\marg{animal}\meta{ascii-art}
%   \end{syntax}
%   adds \anml\ to the known animals. \meta{ascii-art} is multi-line verbatim
%   and therefore should be delimited either by matching braces or by anything
%   that works for \cs{verb}. If the star is given \anml\ is the new default.
%   One space is added to the begin of \anml\ (compensating the opening symbol).
%   For example, |snowman| is added with:\\[1ex]
%   \begin{minipage}{\linewidth}
%\begin{verbatim}
% \AddAnimal{snowman}
% {  \
%     \ _[_]_
%        (")
%     >-( : )-<
%      (__:__)}
%\end{verbatim}
%   \end{minipage}
%   It is not checked whether the animal already exists, you could therefore
%   redefine existing animals with this macro.\\
%   The symbols signalizing the speech (in the |snowman| example above the two
%   backslashes) should at most be used in the first three lines, as they get
%   replaced by |O| and |o| for \cs{duckthink}. They also shouldn't be preceded
%   by anything other than a space in that line.
% \end{function}^^A<<<
%
% \begin{function}{\AddColoredAnimal}^^A>>>
%   \begin{syntax}
%     \cs{AddColoredAnimal}\meta{*}\marg{animal}\meta{ascii-art}
%   \end{syntax}
%   It does the same as \cs{AddAnimal} but allows three different colouring
%   syntaxes. You can use \cs{textcolor} in the \meta{ascii-art} with the syntax
%   \texttt{\cs{textcolor}\marg{color}\marg{text}}. Note that you can't
%   use braces in the arguments of \cs{textcolor}.\\
%   You can also use a delimited \cs{color} of the form
%   \texttt{\cs{bgroup}\cs{color}\marg{color}\meta{text}\cs{egroup}}, a space
%   after that |\egroup| will be considered a space in the output, you don't
%   have to leave a space after the |\egroup| (so
%   |\bgroup\color{red}RedText\egroupOtherText| is valid syntax). You can't nest
%   delimited \cs{color}s.\\
%   Also you can use an undelimited \cs{color}. It affects anything until the
%   end of the current line (or, if used inside of the \meta{text} of an
%   delimited \cs{color}, anything until the end of that delimited \cs{color}'s
%   \meta{text}). The syntax would be \cs{color}\marg{color}.\\
%   The package doesn't load anything providing those colouring commands for you
%   and it doesn't provide any coloured animals. The parsing is done using
%   regular expressions provided by \LaTeX3. It is therefore slower than the
%   normal \cs{AddAnimal}.
% \end{function}^^A<<<
%
%^^A<<<
%
% \subsubsection{Options}\label{sec:options}^^A>>>
% \marginpar
%   {%
%     \vspace*{-2em}\tiny
%     \ducksay[pig,MSG=\footnotesize,align=t]{Options.\\For every occasion}%
%   }
% The following options are available independent on the used code variant (the
% value of the |version| key). They might be used as package options -- unless
% otherwise specified -- or used in the macros \cs{DucksayOptions}, \cs{ducksay}
% and \cs{duckthink} -- again unless otherwise specified. Some options might be
% accessible in both code variants but do slightly different things. If that's
% the case they will be explained in \autoref{sec:options1} and
% \autoref{sec:options2} for |version| 1 and 2, respectively.
% \begin{options}
%   \item[version=\meta{number}]
%     With this you can choose the code variant to be used. Currently |1| and
%     |2| are available. This can be set only during package load time. For a
%     dedicated description of each version look into \autoref{sec:v1} and
%     \autoref{sec:v2}. The package author would choose |version=2|, the other
%     version is mostly for legacy reasons. The default is |2|.
%   \item[\anml] 
%     One of the animals listed in \autoref{sec:animals} or any of the ones
%     added with \cs{AddAnimal}. Not useable as package option. Also don't use
%     it in \cs{DucksayOptions}, it'll break the default animal selection.
%   \item[animal=\anml]
%     Locally sets the default animal. Note that \cs{ducksay} and \cs{duckthink}
%     do digest their options inside of a group, so it just results in
%     a longer alternative to the use of \anml\ if used in their options.
%   \item[ligatures=\meta{token list}]
%     each token you don't want to form ligatures during \cs{AddAnimal} should
%     be contained in this list. All of them get enclosed by grouping |{| and
%     |}| so that they can't form ligatures. Giving no argument (or an empty
%     one) might enhance compilation speed by disabling this replacement. The
%     formation of ligatures was only observed in combination with
%     \verb|\usepackage[T1]{fontenc}| by the author of this package. Therefore
%     giving the option |ligatures| without an argument might enhance the
%     compilation speed for you without any drawbacks. Initially this is set to
%     \texttt{\csuse{\detokenize{l_ducksay_ligatures_tl}}}.\\
%     \textbf{Note:} In earlier releases this option's expected argument was a
%     regular expression. This means that this option is not fully
%     downward compatible with older versions. The speed gain however seems
%     worth it (and I hope the affected documents are few).
%   \item[add-think=\meta{bool}]
%     by default the animals for \cs{duckthink} are not created during package
%     load time, but only when they are really used -- but then they are created
%     globally so it just has to be done once. This is done because they rely on
%     a rather slow regular expression. If you set this key to |true| each
%     \cs{AddAnimal} will also create the corresponding \cs{duckthink} variant
%     immediately.
% \end{options}
%^^A<<<
%
% \SetVersion{1}%
% \subsection{Version 1}\label{sec:v1}
%
% \subsubsection{Introduction}
%
% This version is included for legacy support (old documents should behave the
% same without any change to them -- except the usage of |version=1| as an
% option. For the bleeding edge version of \pkg{ducksay} skip this subsection
% and read \autoref{sec:v2}.
%
% \subsubsection{Macros}\label{sec:macros1}^^A>>>
% \marginpar
%   {%
%     \rlap
%       {%
%         \tiny
%         \ducksay[yoda,MSG=\footnotesize,align=t]{Use those, you might}%
%       }%
%   }
% The following is the description of macros which differ in behaviour from
% those of version 2.
%
% \begin{function}{\ducksay}^^A>>>
%   \begin{syntax}
%     \cs{ducksay}\oarg{options}\marg{message}
%   \end{syntax}
%   options might include any of the options described in \autoref{sec:options}
%   and \autoref{sec:options1} if not otherwise specified. Prints an \anml\
%   saying \msg. \msg\ is not read in verbatim. Multi-line \msg s are possible
%   using |\\|. |\\| should not be contained in a macro definition but at
%   toplevel. Else use the option |ht|.
% \end{function}^^A<<<
%
% \begin{function}{\duckthink}^^A>>>
%   \begin{syntax}
%     \cs{duckthink}\oarg{options}\marg{message}
%   \end{syntax}
%   options might include any of the options described in \autoref{sec:options}
%   and \autoref{sec:options1} if not otherwise specified. Prints an \anml\
%   thinking \msg. \msg\ is not read in verbatim. It is implemented using
%   regular expressions replacing a |\| which is only preceded by |\s*| in the
%   first three lines with |O| and |o|. It is therefore slower than
%   \cs{ducksay}. Multi-line \msg s are possible using |\\|. |\\| should not be
%   contained in a macro definition but at toplevel. Else use the option |ht|.
% \end{function}^^A<<<
%^^A<<<
%
% \subsubsection{Options}\label{sec:options1}^^A>>>
% \marginpar
%   {%
%     \vspace*{-2em}\tiny
%     \hfill
%     \ducksay[hedgehog,MSG=\footnotesize,align=t]{Everyone likes\\options}%
%   }
% The following options are available to \cs{ducksay}, \cs{duckthink}, and
% \cs{DucksayOptions} and if not otherwise specified also as package options:
% \begin{options}
%   \item[bubble=\meta{code}]
%     use \meta{code} in a group right before the bubble (for font switches).
%     Might be used as a package option but not all control sequences work out
%     of the box there.
%   \item[body=\meta{code}]
%     use \meta{code} in a group right before the body (meaning the \anml).
%     Might be used as a package option but not all control sequences work out
%     of the box there. E.g.\@ to right-align the \anml\ to the bubble, use
%     \verb|body=\hfill|.
%   \item[align=\meta{valign}]
%     use \meta{valign} as the vertical alignment specifier given to the
%     \env{tabular} which is around the contents of \cs{ducksay} and
%     \cs{duckthink}.
%   \item[msg-align=\meta{halign}]
%     use \meta{halign} for alignment of the rows of multi-line \msg s. It
%     should match a \texttt{tabular} column specifier. Default is |l|. It only
%     affects the contents of the speech bubble not the bubble.
%   \item[rel-align=\meta{column}]
%     use \meta{column} for alignment of the bubble and the body. It should
%     match a \env{tabular} column specifier. Default is |l|.
%   \item[wd=\meta{count}]
%     in order to detect the width the \msg\ is expanded. This might not work
%     out for some commands (e.g.\@ \cs{url} from \pkg{hyperref}). If you
%     specify the width using |wd| the \msg\ is not expanded and
%     therefore the command \emph{might} work out. \meta{count} should be the
%     character count.
%   \item[ht=\meta{count}]
%     you might explicitly set the height (the row count) of the \msg. This only
%     has an effect if you also specify |wd|.
% \end{options}
%^^A<<<
%
% \subsubsection{Defects}^^A>>>
% \begingroup
%   \reversemarginpar
%   \marginpar
%     {\tiny\hfill\ducksay[frog,MSG=\footnotesize,align=t]{Ohh, no!}}
% \endgroup
% \begin{itemize}
%   \item no automatic line wrapping
% \end{itemize}^^A<<<
%
% \SetVersion{2}%
% \subsection{Version 2}\label{sec:v2}
% \marginpar^^A>>>
%   {
%     \fontsize{3.5pt}{3.5pt}\selectfont
%     \ducksay[unicorn,MSG=\footnotesize,align=t]{Here's all the good stuff!}
%   }^^A<<<
%
% \subsubsection{Introduction}^^A>>>
%
% Version 2 is the current version of \pkg{ducksay}. It features automatic line
% wrapping (if you specify a fixed width) and in general more options (with some
% nasty argument parsing).
%
% If you're already used to version 1 you should note one important thing: You
% should only specify the |version|, the |ligatures| and |add-think| during
% package load time as arguments to \cs{usepackage}. The other keys might not
% work or do unintended things and only don't throw errors or warnings because
% of the legacy support of version 1.
%
%^^A<<<
%
% \subsubsection{Macros}\label{sec:macros2}^^A>>>
% \marginpar^^A>>>
%   {
%     \tiny
%     \ducksay[duck-family,MSG=\footnotesize,align=t,body-mirrored]
%       {Look at those, kids!}
%   }^^A<<<
%
% The following is the description of macros which differ in behaviour from
% those of version 1.
%
% \begin{function}{\ducksay}^^A>>>
%   \begin{syntax}
%     \cs{ducksay}\oarg{options}\marg{message}
%   \end{syntax}
%   options might include any of the options described in \autoref{sec:options}
%   and \autoref{sec:options2} if not otherwise specified. Prints an \anml\
%   saying \msg.\\
%   The \msg\ can be read in in four different ways. For an explanation of the
%   \msg\ reading see the description of the |arg| key in
%   \autoref{sec:options2}.\\
%   The height and width of the message is determined by measuring its
%   dimensions and the bubble will be set accordingly. The box surrounding the
%   message will be placed both horizontally and vertically centred inside of
%   the bubble. The output utilizes \LaTeX3's coffin mechanism described in
%   \href{https://ctan.org/pkg/l3kernel}{\file{interface3.pdf}} and the
%   documentation of \href{https://ctan.org/pkg/xcoffins}{\pkg{xcoffins}}.
% \end{function}^^A<<<
%
% \begin{function}{\duckthink}^^A>>>
%   \begin{syntax}
%     \cs{duckthink}\oarg{options}\marg{message}
%   \end{syntax}
%   The only difference to \cs{ducksay} is that in \cs{duckthink} the \anml s
%   think the \msg\ and don't say it.\\
%   It is implemented using regular expressions replacing a |\| which is only
%   preceded by |\s*| (any number of space tokens) in the first three lines with
%   |O| and |o|. It's first use per \anml\ might therefore be slower than
%   \cs{ducksay} depending on the |add-think| key (see its description in
%   \autoref{sec:options}).
% \end{function}^^A<<<
%
%^^A<<<
%
% \subsubsection{Options}\label{sec:options2}^^A>>>
% \marginpar^^A>>>
%   {
%     \tiny
%     \hfill\ducksay[small-rabbit,MSG=\footnotesize,align=t]
%       {Fast, use options!}
%   }^^A<<<
% In version 2 the following options are available. Keep in mind that you
% shouldn't use them during package load time but in the arguments of
% \cs{ducksay}, \cs{duckthink} or \cs{DucksayOptions}.
% \begin{options}
%   \item[arg=\meta{choice}]
%     specifies how the \msg\ argument of \cs{ducksay} and \cs{duckthink} should
%     be read in. Available options are |box|, |tab| and |tab*|:
%     \begin{options}
%       \item[box]
%         the argument is read in either as a \cs{hbox} or a \cs{vbox} (the
%         latter if a fixed width is specified with either |wd| or |wd*|). Note
%         that in this mode any arguments relying on category code changes like
%         e.g.\@ \cs{verb} will work (provided that you don't use \cs{ducksay} or
%         \cs{duckthink} inside of an argument of another macro of course).
%       \item[tab]
%         the argument is read in as the contents of a \env{tabular}. Note that
%         in this mode any arguments relying on category code changes like
%         e.g.\@ \cs{verb} will \emph{not} work. This mode comes closest to the
%         behaviour of version 1 of \pkg{ducksay}.
%       \item[tab*]
%         the argument is read in as the contents of a \env{tabular}. However it
%         is read in verbatim and uses \cs{scantokens} to rescan the argument.
%         Note that in this mode any arguments relying on category code changes
%         like e.g.\@ \cs{verb} will work. You can't use \cs{ducksay} or
%         \cs{duckthink} as an argument to another macro in this mode however.
%     \end{options}
%   \item[b]
%     shortcut for |out-v=b|.
%   \item[body=\meta{font}]
%     add \meta{font} to the font definitions in use to typeset the \anml's
%     body.
%   \item[body*=\meta{font}]
%     clear any definitions previously made (including the package default) and
%     set the font definitions in use to typeset the \anml's body to
%     \meta{font}. The package default is \cs{verbatim@font}. In addition
%     \cs{frenchspacing} will always be used prior to the defined \meta{font}.
%   \item[body-align=\meta{choice}]
%     sets the relative alignment of the \anml\ to the \msg. Possible choices
%     are |l|, |c| and |r|. For |l| the \anml\ is flushed to the left of the
%     \msg, for |c| it is centred and for |r| it is flushed right. More fine
%     grained control over the alignment can be obtained with the keys
%     |msg-to-body|, |body-to-msg|, |body-x| and |body-y|. Package default is
%     |l|.
%   \item[body-mirrored=\meta{bool}]
%     if set true the \anml\ will be mirrored along its vertical centre axis.
%     Package default is |false|. If you set it |true| you'll most likely need
%     to manually adjust the alignment of the body with one or more of the
%     keys |body-align|, |body-to-msg|, |msg-to-body|, |body-x| and |body-y|.
%   \item[body-to-msg=\meta{pole}]
%     defines the horizontal coffin \meta{pole} to be used for the placement of
%     the \anml\ beneath the \msg. See \PolesInfo.
%   \item[body-x=\meta{dimen}]
%     defines a horizontal offset of \meta{dimen} length of the \anml\ from its
%     placement beneath the \msg.
%   \item[body-y=\meta{dimen}]
%     defines a vertical offset of \meta{dimen} length of the \anml\ from its
%     placement beneath the \msg.
%   \item[bubble=\meta{font}]
%     add \meta{font} to the font definitions in use to typeset the bubble. This
%     does not affect the \msg\ only the bubble put around it.
%   \item[bubble*=\meta{font}]
%     clear any definitions previously made (including the package default) and
%     set the font definitions in use to typeset the bubble to \meta{font}. This
%     does not affect the \msg\ only the bubble put around it. The package
%     default is \cs{verbatim@font}.
%   \item[bubble-bot-kern=\meta{dimen}]
%     specifies a vertical offset of the placement of the lower border of the
%     bubble from the bottom of the left and right borders.
%   \item[bubble-delim-left-1=\meta{token list}]
%     the left delimiter used if only one line of delimiters is needed. Package
%     default is |(|.
%   \item[bubble-delim-left-2=\meta{token list}]
%     the upper most left delimiter used if more than one line of delimiters is
%     needed. Package default is |/|.
%   \item[bubble-delim-left-3=\meta{token list}]
%     the left delimiters used to fill the gap if more than two lines of
%     delimiters are needed. Package default is \verb+|+.
%   \item[bubble-delim-left-4=\meta{token list}]
%     the lower most left delimiters used if more than one line of delimiters is
%     needed. Package default is |\|.
%   \item[bubble-delim-right-1=\meta{token list}]
%     the right delimiter used if only one line of delimiters is needed. Package
%     default is |)|.
%   \item[bubble-delim-right-2=\meta{token list}]
%     the upper most right delimiter used if more than one line of delimiters is
%     needed. Package default is |\|.
%   \item[bubble-delim-right-3=\meta{token list}]
%     the right delimiters used to fill the gap if more than two lines of
%     delimiters are needed. Package default is \verb+|+.
%   \item[bubble-delim-right-4=\meta{token list}]
%     the lower most right delimiters used if more than one line of delimiters
%     is needed. Package default is |/|.
%   \item[bubble-delim-top=\meta{token list}]
%     the delimiter used to create the top and bottom border of the bubble. The
%     package default is |{-}| (the braces are important to suppress ligatures
%     here).
%   \item[bubble-side-kern=\meta{dimen}]
%     specifies the kerning used to move the sideways delimiters added to fill
%     the gap for more than two lines of bubble height. (the left one is moved
%     to the left, the right one to the right)
%   \item[bubble-top-kern=\meta{dimen}]
%     specifies a vertical offset of the placement of the upper border of the
%     bubble from the top of the left and right borders.
%   \item[c]
%     shortcut for |out-v=vc|.
%   \item[col=\meta{column}]
%     specifies the used column specifier used for the \msg\ enclosing
%     \env{tabular} for |arg=tab| and |arg=tab*|. Has precedence over
%     |msg-align|. You can also use more than one column this way:
%     |\ducksay[arg=tab,col=cc]{ You & can \\ do & it }| would be valid syntax.
%   \item[hpad=\meta{count}]
%     Add \meta{count} times more |bubble-delim-top| instances than necassary to
%     the upper and lower border of the bubble. Package default is 2.
%   \item[ht=\meta{count}]
%     specifies a minimum height (in lines) of the \msg. The lines' count is
%     that of the needed lines of the horizontal bubble delimiters. If the
%     count of the actually needed lines is smaller than the specified
%     \meta{count}, \meta{count} lines will be used. Else the required lines
%     will be used.
%   \item[ignore-body=\meta{bool}]
%     If set |true| the \anml's body will be added to the output but it will not
%     contribute to the bounding box (so will not take up any space).
%   \item[msg=\meta{font}]
%     add \meta{font} to the font definitions in use to typeset the \msg.
%   \item[msg*=\meta{font}]
%     clear any definitions previously made (including the package default) and
%     set the font definitions in use to typeset the \msg\ to \meta{font}. The
%     package default is \cs{verbatim@font}.
%   \item[MSG=\meta{font}]
%     same as \texttt{msg=\meta{font}, bubble=\meta{font}}.
%   \item[MSG*=\meta{font}]
%     same as \texttt{msg*=\meta{font}, bubble*=\meta{font}}.
%   \item[msg-align=\meta{choice}]
%     specifies the alignment of the \msg. Possible values are |l| for flushed
%     left, |c| for centred, |r| for flushed right and |j| for justified. If
%     |arg=tab| or |arg=tab*| the |j| choice is only available for fixed width
%     contents. Package default is |l|.
%   \item[msg-align-c=\meta{token list}]
%     set the \meta{token list} which is responsible to typeset the message
%     centred if the option |msg-align=c| is used. It is used independent of the
%     |arg| key. For |arg=tab| and |arg=tab*| the macro \cs{arraybackslash}
%     provided by \pkg{array} is used afterwards. The package default is
%     |\centering|. It might be useful if you want to use \pkg{ragged2e}'s
%     \cs{Centering} for example.
%   \item[msg-align-j=\meta{token list}]
%     set the \meta{token list} which is responsible to typeset the message
%     justified if the option |msg-align=j| is used. It is used independent of
%     the |arg| key. For |arg=tab| and |arg=tab*| the macro \cs{arraybackslash}
%     provided by \pkg{array} is used afterwards. The package default is
%     empty as justification is the default behaviour of contents of a |p|
%     column and of a \cs{vbox}. It might be useful if you want to use
%     \pkg{ragged2e}'s \cs{justifying} for example.
%   \item[msg-align-l=\meta{token list}]
%     set the \meta{token list} which is responsible to typeset the message
%     flushed left if the option |msg-align=l| is used. It is used independent
%     of the |arg| key. For |arg=tab| and |arg=tab*| the macro
%     \cs{arraybackslash} provided by \pkg{array} is used afterwards. The
%     package default is |\raggedright|. It might be useful if you want to use
%     \pkg{ragged2e}'s \cs{RaggedRight} for example.
%   \item[msg-align-r=\meta{token list}]
%     set the \meta{token list} which is responsible to typeset the message
%     flushed right if the option |msg-align=r| is used. It is used independent
%     of the |arg| key. For |arg=tab| and |arg=tab*| the macro
%     \cs{arraybackslash} provided by \pkg{array} is used afterwards. The
%     package default is |\raggedleft|. It might be useful if you want to use
%     \pkg{ragged2e}'s \cs{RaggedLeft} for example.
%   \item[msg-to-bubble=\meta{pole}]
%     defines the horizontal coffin \meta{pole} to be used as the reference
%     point for the placement of the \anml\ beneath the \msg. See \PolesInfo.
%   \item[none=\meta{bool}]
%     One could say this is a special animal. If |true| no animal body will be
%     used (resulting in just the speech bubble). Package default is of course
%     |false|.
%   \item[out-h=\meta{pole}]
%     defines the horizontal coffin \meta{pole} to be used as the anchor point
%     for the print out of the complete result of \cs{ducksay} and
%     \cs{duckthink}. See \PolesInfo.
%   \item[out-v=\meta{pole}]
%     defines the vertical coffin \meta{pole} to be used as the anchor point for
%     the print out of the complete result of \cs{ducksay} and \cs{duckthink}.
%     See \PolesInfo.
%   \item[out-x=\meta{dimen}]
%     specifies an additional horizontal offset of the print out of the complete
%     result of \cs{ducksay} and \cs{duckthink}.
%   \item[out-y=\meta{dimen}]
%     specifies an additional vertical offset of the print out of the complete
%     result of \cs{ducksay} and \cs{duckthink}
%   \item[t]
%     shortcut for |out-v=t|.
%   \item[vpad=\meta{count}]
%     add \meta{count} to the lines used for the bubble, resulting in
%     \meta{count} more lines than necessary to enclose the \msg\ inside of the
%     bubble.
%   \item[wd=\meta{count}]
%     specifies the width of the \msg\ to be fixed to \meta{count} times the
%     width of an upper case M in the \msg's font declaration. A value smaller
%     than 0 is considered deactivated, else the width is considered as fixed.
%     For a fixed width the argument of \cs{ducksay} and \cs{duckthink} is read
%     in as a \cs{vbox} for |arg=box| and the column definition uses a |p|-type
%     column for |arg=tab| and |arg=tab*|. If both |wd| is not smaller than 0
%     and |wd*| is not smaller than 0pt, |wd*| will take precedence.
%   \item[wd*=\meta{dimen}]
%     specifies the width of the \msg\ to be fixed to \meta{dimen}. A value
%     smaller than 0pt is considered deactivated, else the width is considered
%     as fixed. For a fixed width the argument of \cs{ducksay} and
%     \cs{duckthink} is read in as a \cs{vbox} for |arg=box| and the column
%     definition uses a |p|-type column for |arg=tab| and |arg=tab*|. If both
%     |wd| is not smaller than 0 and |wd*| is not smaller than 0pt, |wd*| will
%     take precedence.
% \end{options}
%
%^^A<<<
%
% \SetVersion{}%
% \subsection{Dependencies}^^A>>>
% \marginpar
%   {%
%     \tiny
%     \rlap
%       {%
%         \ducksay
%           [
%             kangaroo,MSG=\footnotesize,align=t
%             ,body-mirrored,body-to-msg=r,msg-to-body=hc
%           ]
%           {We rely on you}%
%       }%
%   }
% The package depends on the two packages \pkg{xparse} and \pkg{l3keys2e}
% and all of their dependencies. Version 2 additionally depends on \pkg{array}.
%^^A<<<
%
% \subsection{Available Animals}\label{sec:animals}^^A>>>
% \marginpar
%   {%
%     \tiny
%     \hfill
%     \makebox[8em][r]
%       {%
%         \ducksay[whale,MSG=\footnotesize,align=t]
%           {I'm the\\new one.}%
%       }%
%   }
% The following animals are provided by this package. I did not create them (but
% altered some), they belong to their original creators.
% \bgroup
% \fontsize{6pt}{6pt}\selectfont
% \parindent=0pt
% \DucksayOptions{MSG=\footnotesize,vpad=0,arg=tab}
% \begin{multicols}{2}
% \availableAnimal{^^A>>>
%   ,duck^^A
%   ,small-duck^^A
%   ,duck-family^^A
%   ,small-rabbit^^A
%   ,squirrel^^A
%   ,cow^^A
%   ,tux^^A
%   ,head-in^^A
%   ,pig^^A
%   ,frog^^A
%   ,snowman^^A
%   ,bunny^^A
%   ,dragon^^A
%   ,sodomized^^A
%   ,hedgehog^^A
%   ,kangaroo^^A
%   ,dog^^A
%   ,rabbit^^A
%   ,snail^^A
%   ,whale^^A
%   ,unicorn^^A
% }\end{multicols}\begin{multicols}{2}
% \availableAnimal{^^A
%   ,r2d2^^A
%   ,vader^^A
%   ,yoda-head^^A
%   ,small-yoda^^A
%   ,yoda^^A
% }^^A<<<
% \end{multicols}
% \egroup
%^^A<<<
%
% \subsection{Miscellaneous}^^A>>>
% \marginpar
%   {%
%     \rlap
%       {%
%         \tiny
%         \ducksay[squirrel,MSG=\footnotesize,align=t]
%           {WTFPL would be a\\better license.}%
%       }%
%   }
% This work  may be  distributed and/or  modified under  the conditions  of the
% \LaTeX\ Project Public License (LPPL),  either version 1.3c  of this license
% or (at your option) any later version.  The latest version of this license is
% in the file:
%   \url{http://www.latex-project.org/lppl.txt}
%
% The package is hosted on \url{https://github.com/Skillmon/ltx_ducksay}, you
% might report bugs there.
%^^A<<<
%
% \end{documentation}^^A<<<
%
% \begin{implementation}^^A>>>
%
% \clearpage
%
% \SetVersion{}\def\DocImp{Implementation}%
% \section{Implementation}^^A>>>
% \marginpar
%   {%
%     \smash
%       {%
%         \tiny
%         \ducksay
%           [vader,MSG=\footnotesize,align=t,arg=tab,msg-align=c]
%           {%
%             Only rebel scum reads\\documentation!\\
%             Join the dark side,\\read the implementation.%
%           }%
%       }%
%   }
%
%^^A main file >>>
%    \begin{macrocode}
%<*pkg>
%    \end{macrocode}
%
% \subsection{Shared between versions}^^A>>>
%
% \subsubsection{Variables}^^A>>>
% \paragraph{Integers}
%    \begin{macrocode}
\int_new:N \l_ducksay_msg_width_int
\int_new:N \l_ducksay_msg_height_int
%    \end{macrocode}
% \paragraph{Sequences}
%    \begin{macrocode}
\seq_new:N \l_ducksay_msg_lines_seq
%    \end{macrocode}
% \paragraph{Token lists}
%    \begin{macrocode}
\tl_new:N \l_ducksay_say_or_think_tl
\tl_new:N \l_ducksay_align_tl
\tl_new:N \l_ducksay_msg_align_tl
\tl_new:N \l_ducksay_animal_tl
\tl_new:N \l_ducksay_body_tl
\tl_new:N \l_ducksay_bubble_tl
\tl_new:N \l_ducksay_tmpa_tl
%    \end{macrocode}
% \paragraph{Boolean}
%    \begin{macrocode}
\bool_new:N \l_ducksay_also_add_think_bool
\bool_new:N \l_ducksay_version_one_bool
\bool_new:N \l_ducksay_version_two_bool
%    \end{macrocode}
% \paragraph{Boxes}
%    \begin{macrocode}
\box_new:N \l_ducksay_tmpa_box
%    \end{macrocode}
%
%^^A<<<
%
% \subsubsection{Regular Expressions}^^A>>>
% Regular expressions for \cs{duckthink}
%    \begin{macrocode}
\regex_const:Nn \c_ducksay_first_regex  { \A(.\s*)\\ }
\regex_const:Nn \c_ducksay_second_regex { \A(.[^\c{null}]*\c{null}\s*)\\ }
\regex_const:Nn \c_ducksay_third_regex  {
  \A(.[^\c{null}]*\c{null}[^\c{null}]*\c{null}\s*)\\ }
\regex_const:Nn \c_ducksay_textcolor_regex
  { \cO(?:\\textcolor\{(.*?)\}\{(.*?)\}) }
\regex_const:Nn \c_ducksay_color_delim_regex
  { \cO(?:\\bgroup\\color\{(.*?)\}(.*)\\egroup) }
\regex_const:Nn \c_ducksay_color_regex
  { \cO(?:\\color\{(.*?)\}) }
%    \end{macrocode}
%^^A<<<
%
% \subsubsection{Messages}^^A>>>
%    \begin{macrocode}
\msg_new:nnn { ducksay } { load-time-only }
  { The~`#1`~key~is~to~be~used~only~during~package~load~time. }
%    \end{macrocode}
%^^A<<<
%
% \subsubsection{Key-value setup}^^A>>>
%    \begin{macrocode}
\keys_define:nn { ducksay }
  {
    ,bubble .tl_set:N      = \l_ducksay_bubble_tl
    ,body   .tl_set:N      = \l_ducksay_body_tl
    ,align  .tl_set:N      = \l_ducksay_align_tl
    ,align  .value_required:n = true
    ,wd     .int_set:N     = \l_ducksay_msg_width_int
    ,wd     .initial:n     = -\c_max_int
    ,wd     .value_required:n = true
    ,ht     .int_set:N     = \l_ducksay_msg_height_int
    ,ht     .initial:n     = -\c_max_int
    ,ht     .value_required:n = true
    ,animal .code:n        =
      { \keys_define:nn { ducksay } { default_animal .meta:n = { #1 } } }
    ,animal .initial:n     = duck
    ,msg-align .tl_set:N   = \l_ducksay_msg_align_tl
    ,msg-align .initial:n  = l
    ,msg-align .value_required:n = true
    ,rel-align .tl_set:N   = \l_ducksay_rel_align_tl
    ,rel-align .initial:n  = l
    ,rel-align .value_required:n = true
    ,ligatures .tl_set:N   = \l_ducksay_ligatures_tl
    ,ligatures .initial:n  = { `<>,'- }
    ,add-think .bool_set:N = \l_ducksay_also_add_think_bool
    ,version   .choice:
    ,version / 1 .code:n   = 
      {
        \bool_set_false:N \l_ducksay_version_two_bool
        \bool_set_true:N  \l_ducksay_version_one_bool
      }
    ,version / 2 .code:n   =
      {
        \bool_set_false:N \l_ducksay_version_one_bool
        \bool_set_true:N  \l_ducksay_version_two_bool
      }
    ,version   .initial:n  = 2
  }
%    \end{macrocode}
%
%    \begin{macrocode}
\ProcessKeysOptions { ducksay }
%    \end{macrocode}
%
% Undefine the load-time-only keys
%    \begin{macrocode}
\keys_define:nn { ducksay }
  {
    version .code:n = \msg_error:nnn { ducksay } { load-time-only } { version }
  }
%    \end{macrocode}
%
%^^A<<<
%
% \subsubsection{Functions}^^A>>>
%
% \paragraph{Generating Variants of External Functions}^^A>>>
%
%    \begin{macrocode}
\cs_generate_variant:Nn \tl_if_eq:nnT { VnT }
%    \end{macrocode}
%^^A<<<
%
% \paragraph{Internal}^^A>>>
%
% \begin{macro}{\ducksay_create_think_animal:n}^^A>>>
%    \begin{macrocode}
\cs_new_protected:Npn \ducksay_create_think_animal:n #1
  {
    \group_begin:
      \tl_set_eq:Nc \l_ducksay_tmpa_tl { g_ducksay_animal_say_#1_tl }
      \regex_replace_once:NnN \c_ducksay_first_regex  { \1O } \l_ducksay_tmpa_tl
      \regex_replace_once:NnN \c_ducksay_second_regex { \1o } \l_ducksay_tmpa_tl
      \regex_replace_once:NnN \c_ducksay_third_regex  { \1o } \l_ducksay_tmpa_tl
      \tl_gset_eq:cN { g_ducksay_animal_think_#1_tl } \l_ducksay_tmpa_tl
    \group_end:
  }
%    \end{macrocode}
% \end{macro}^^A<<<
%
% \begin{macro}{\ducksay_replace_verb_newline:Nn}^^A>>>
%    \begin{macrocode}
\cs_new_protected:Npx \ducksay_replace_verb_newline:Nn #1 #2
  {
    \tl_replace_all:Nnn #1 { \char_generate:nn { 13 } { 12 } } { #2 }
  }
%    \end{macrocode}
% \end{macro}^^A<<<
%
% \begin{macro}{\ducksay_replace_verb_newline_newline:Nn}^^A>>>
%    \begin{macrocode}
\cs_new_protected:Npx \ducksay_replace_verb_newline_newline:Nn #1 #2
  {
    \tl_replace_all:Nnn #1
      { \char_generate:nn { 13 } { 12 } \char_generate:nn { 13 } { 12 } } { #2 }
  }
%    \end{macrocode}
% \end{macro}^^A<<<
%
% \begin{macro}{\ducksay_process_verb_newline:nnn}^^A>>>
%    \begin{macrocode}
\cs_new_protected:Npn \ducksay_process_verb_newline:nnn #1 #2 #3
  {
    \tl_set:Nn \ProcessedArgument { #3 }
    \ducksay_replace_verb_newline_newline:Nn \ProcessedArgument { #2 }
    \ducksay_replace_verb_newline:Nn \ProcessedArgument { #1 }
  }
%    \end{macrocode}
% \end{macro}^^A<<<
%
% \begin{macro}{\ducksay_add_animal_inner:nn}^^A>>>
%    \begin{macrocode}
\cs_new_protected:Npn \ducksay_add_animal_inner:nn #1 #2
  {
    \tl_set:Nn \l_ducksay_tmpa_tl { \ #2 }
    \tl_map_inline:Nn \l_ducksay_ligatures_tl
      { \tl_replace_all:Nnn \l_ducksay_tmpa_tl { ##1 } { { ##1 } } }
    \ducksay_replace_verb_newline:Nn \l_ducksay_tmpa_tl { \tabularnewline\null }
    \tl_gset_eq:cN { g_ducksay_animal_say_#1_tl } \l_ducksay_tmpa_tl
    \keys_define:nn { ducksay }
      {
        #1 .code:n =
          {
            \tl_if_exist:cF
              { g_ducksay_animal_ \l_ducksay_say_or_think_tl _#1_tl }
              { \ducksay_create_think_animal:n { #1 } }
            \tl_set_eq:Nc \l_ducksay_animal_tl
              { g_ducksay_animal_ \l_ducksay_say_or_think_tl _#1_tl }
          }
      }
  }
%    \end{macrocode}
% \end{macro}^^A<<<
%
%^^A<<<
%
% \paragraph{Document level}^^A>>>
%
% \begin{macro}{\DefaultAnimal}^^A>>>
%    \begin{macrocode}
\NewDocumentCommand \DefaultAnimal { m }
  {
    \keys_define:nn { ducksay } { default_animal .meta:n = { #1 } }
  }
%    \end{macrocode}
% \end{macro}^^A<<<
%
% \begin{macro}{\DucksayOptions}^^A>>>
%    \begin{macrocode}
\NewDocumentCommand \DucksayOptions { m }
  {
    \keys_set:nn { ducksay } { #1 }
  }
%    \end{macrocode}
% \end{macro}^^A<<<
%
% \begin{macro}{\AddAnimal}^^A>>>
%    \begin{macrocode}
\NewDocumentCommand \AddAnimal { s m +v }
  {
    \ducksay_add_animal_inner:nn { #2 } { #3 }
    \bool_if:NT \l_ducksay_also_add_think_bool
      { \ducksay_create_think_animal:n { #2 } }
    \IfBooleanT{#1}
      { \keys_define:nn { ducksay } { default_animal .meta:n = { #2 } } }
  }
%    \end{macrocode}
% \end{macro}^^A<<<
%
% \begin{macro}{\AddColoredAnimal}^^A>>>
%    \begin{macrocode}
\NewDocumentCommand \AddColoredAnimal { s m +v }
  {
    \ducksay_add_animal_inner:nn { #2 } { #3 }
    \regex_replace_all:Nnc \c_ducksay_color_delim_regex
      { \c{bgroup}\c{color}\cB\{\1\cE\}\2\c{egroup} }
      { g_ducksay_animal_say_#2_tl }
    \regex_replace_all:Nnc \c_ducksay_color_regex
      { \c{color}\cB\{\1\cE\} }
      { g_ducksay_animal_say_#2_tl }
    \regex_replace_all:Nnc \c_ducksay_textcolor_regex
      { \c{textcolor}\cB\{\1\cE\}\cB\{\2\cE\} }
      { g_ducksay_animal_say_#2_tl }
    \bool_if:NT \l_ducksay_also_add_think_bool
      { \ducksay_create_think_animal:n { #2 } }
    \IfBooleanT{#1}
      { \keys_define:nn { ducksay } { default_animal .meta:n = { #2 } } }
  }
%    \end{macrocode}
% \end{macro}^^A<<<
%
%^^A<<<
%
%^^A<<<
%
% \subsubsection{Load the Correct Version and the Animals}^^A>>>
%    \begin{macrocode}
\bool_if:NT \l_ducksay_version_one_bool
  { \file_input:n { ducksay.code.v1.tex } }
\bool_if:NT \l_ducksay_version_two_bool
  { \file_input:n { ducksay.code.v2.tex } }
%    \end{macrocode}
%
%    \begin{macrocode}
\ExplSyntaxOff
\input{ducksay.animals.tex}
%    \end{macrocode}
%^^A<<<
%
%^^A<<<
%
%    \begin{macrocode}
%</pkg>
%    \end{macrocode}
%^^A<<<
%
% \SetVersion{1}%
% \subsection{Version 1}^^A>>>
%    \begin{macrocode}
%<*code.v1>
%    \end{macrocode}
%
% \subsubsection{Functions}^^A>>>
%
% \paragraph{Internal}^^A>>>
%
% \begin{macro}{\ducksay_longest_line:n}^^A>>>
%   Calculate the length of the longest line
%    \begin{macrocode}
\cs_new:Npn \ducksay_longest_line:n #1
  {
    \int_incr:N \l_ducksay_msg_height_int
    \exp_args:NNx \tl_set:Nn \l_ducksay_tmpa_tl { #1 }
    \regex_replace_all:nnN { \s } { \c { space } } \l_ducksay_tmpa_tl
    \int_set:Nn \l_ducksay_msg_width_int
      {
        \int_max:nn
          { \l_ducksay_msg_width_int } { \tl_count:N \l_ducksay_tmpa_tl }
      }
  }
%    \end{macrocode}
% \end{macro}^^A<<<
%
% \begin{macro}{\ducksay_open_bubble:}^^A>>>
%   Draw the opening bracket of the bubble
%    \begin{macrocode}
\cs_new:Npn \ducksay_open_bubble:
  {
    \begin{tabular}{@{}l@{}}
      \null\\
      \int_compare:nNnTF { \l_ducksay_msg_height_int } = { 1 } { ( }
        {
          /
          \int_step_inline:nnn
            { 3 } { \l_ducksay_msg_height_int } { \\\kern-0.2em| }
          \\\detokenize{\ }
        }
      \\[-1ex]\null
    \end{tabular}
    \begin{tabular}{@{}l@{}}
      _\\
      \int_step_inline:nnn { 2 } { \l_ducksay_msg_height_int } { \\ } \\[-1ex]
      \mbox { - }
    \end{tabular}
  }
%    \end{macrocode}
% \end{macro}^^A<<<
%
% \begin{macro}{\ducksay_close_bubble:}^^A>>>
%   Draw the closing bracket of the bubble
%    \begin{macrocode}
\cs_new:Npn \ducksay_close_bubble:
  {
    \begin{tabular}{@{}l@{}}
      _\\
      \int_step_inline:nnn { 2 } { \l_ducksay_msg_height_int } { \\ } \\[-1ex]
      { - }
    \end{tabular}
    \begin{tabular}{@{}r@{}}
      \null\\
      \int_compare:nNnTF { \l_ducksay_msg_height_int } = { 1 }
        { ) }
        {
          \detokenize {\ }
          \int_step_inline:nnn
            { 3 } { \l_ducksay_msg_height_int } { \\|\kern-0.2em }
          \\/
        }
      \\[-1ex]\null
    \end{tabular}
  }
%    \end{macrocode}
% \end{macro}^^A<<<
%
% \begin{macro}{\ducksay_print_msg:nn}^^A>>>
%   Print out the message
%    \begin{macrocode}
\cs_new:Npn \ducksay_print_msg:nn #1 #2
  {
    \begin{tabular}{@{} #2 @{}}
      \int_step_inline:nn { \l_ducksay_msg_width_int } { _ } \\
      #1\\[-1ex]
      \int_step_inline:nn { \l_ducksay_msg_width_int } { { - } }
    \end{tabular}
  }
\cs_generate_variant:Nn \ducksay_print_msg:nn { nV }
%    \end{macrocode}
% \end{macro}^^A<<<
%
% \begin{macro}{\ducksay_print:nn}^^A>>>
%   Print out the whole thing
%    \begin{macrocode}
\cs_new:Npn \ducksay_print:nn #1 #2
  {
    \int_compare:nNnTF { \l_ducksay_msg_width_int } < { 0 }
      {
        \int_zero:N \l_ducksay_msg_height_int
        \seq_set_split:Nnn \l_ducksay_msg_lines_seq { \\ } { #1 }
        \seq_map_function:NN \l_ducksay_msg_lines_seq \ducksay_longest_line:n
      }
      {
        \int_compare:nNnT { \l_ducksay_msg_height_int } < { 0 }
          {
            \regex_count:nnN { \c { \\ } } { #1 } \l_ducksay_msg_height_int
            \int_incr:N \l_ducksay_msg_height_int
          }
      }
    \group_begin:
      \frenchspacing
      \verbatim@font
      \@noligs
      \begin{tabular}[\l_ducksay_align_tl]{@{}#2@{}}
        \l_ducksay_bubble_tl
        \begin{tabular}{@{}l@{}}
          \ducksay_open_bubble:
          \ducksay_print_msg:nV { #1 } \l_ducksay_msg_align_tl
          \ducksay_close_bubble:
        \end{tabular}\\
        \l_ducksay_body_tl
        \begin{tabular}{@{}l@{}}
          \l_ducksay_animal_tl
        \end{tabular}
      \end{tabular}
    \group_end:
  }
\cs_generate_variant:Nn \ducksay_print:nn { nV }
%    \end{macrocode}
% \end{macro}^^A<<<
%
% \begin{macro}{\ducksay_prepare_say_and_think:n}^^A>>>
%   Reset some variables
%    \begin{macrocode}
\cs_new:Npn \ducksay_prepare_say_and_think:n #1
  {
    \int_set:Nn \l_ducksay_msg_width_int  { -\c_max_int }
    \int_set:Nn \l_ducksay_msg_height_int { -\c_max_int }
    \keys_set:nn { ducksay } { #1 }
    \tl_if_empty:NT \l_ducksay_animal_tl
      { \keys_set:nn { ducksay } { default_animal } }
  }
%    \end{macrocode}
% \end{macro}^^A<<<
%^^A<<<
%
% \paragraph{Document level}^^A>>>
%
% \begin{macro}{\ducksay}^^A>>>
%    \begin{macrocode}
\NewDocumentCommand \ducksay { O{} m }
  {
    \group_begin:
      \tl_set:Nn \l_ducksay_say_or_think_tl { say }
      \ducksay_prepare_say_and_think:n { #1 }
      \ducksay_print:nV { #2 } \l_ducksay_rel_align_tl
    \group_end:
  }
%    \end{macrocode}
% \end{macro}^^A<<<
%
% \begin{macro}{\duckthink}^^A>>>
%    \begin{macrocode}
\NewDocumentCommand \duckthink { O{} m }
  {
    \group_begin:
      \tl_set:Nn \l_ducksay_say_or_think_tl { think }
      \ducksay_prepare_say_and_think:n { #1 }
      \ducksay_print:nV { #2 } \l_ducksay_rel_align_tl
    \group_end:
  }
%    \end{macrocode}
% \end{macro}^^A<<<
%^^A<<<
%
%^^A<<<
%
%    \begin{macrocode}
%</code.v1>
%    \end{macrocode}^^A<<<
%
% \SetVersion{2}%
% \subsection{Version 2}^^A>>>
%    \begin{macrocode}
%<*code.v2>
%    \end{macrocode}
%
% Load the additional dependencies of version 2.
%    \begin{macrocode}
\RequirePackage{array}
%    \end{macrocode}
%
% \subsubsection{Messages}^^A>>>
%    \begin{macrocode}
\msg_new:nnn { ducksay } { justify~unavailable }
  {
    Justified~content~is~not~available~for~tabular~argument~mode~without~fixed~
    width.~`l`~column~is~used~instead.
  }
\msg_new:nnn { ducksay } { unknown~message~alignment }
  {
    The~specified~message~alignment~`\exp_not:n { #1 }`~is~unknown.~
    `l`~is~used~as~fallback.
  }
%    \end{macrocode}
%^^A<<<
%
% \subsubsection{Variables}^^A>>>
%
% \paragraph{Token Lists}
%    \begin{macrocode}
\tl_new:N \l_ducksay_msg_align_vbox_tl
%    \end{macrocode}
%
% \paragraph{Boxes}
%    \begin{macrocode}
\box_new:N \l_ducksay_msg_box
%    \end{macrocode}
%
% \paragraph{Bools}
%    \begin{macrocode}
\bool_new:N \l_ducksay_eat_arg_box_bool
\bool_new:N \l_ducksay_eat_arg_tab_verb_bool
\bool_new:N \l_ducksay_mirrored_body_bool
%    \end{macrocode}
%
% \paragraph{Coffins}
%    \begin{macrocode}
\coffin_new:N \l_ducksay_body_coffin
\coffin_new:N \l_ducksay_bubble_close_coffin
\coffin_new:N \l_ducksay_bubble_open_coffin
\coffin_new:N \l_ducksay_bubble_top_coffin
\coffin_new:N \l_ducksay_msg_coffin
%    \end{macrocode}
%
% \paragraph{Dimensions}
%    \begin{macrocode}
\dim_new:N \l_ducksay_hpad_dim
\dim_new:N \l_ducksay_bubble_bottom_kern_dim
\dim_new:N \l_ducksay_bubble_top_kern_dim
\dim_new:N \l_ducksay_msg_width_dim
%    \end{macrocode}
%
%^^A<<<
%
% \subsubsection{Options}^^A>>>
%
%    \begin{macrocode}
\keys_define:nn { ducksay }
  {
    ,arg .choice:
    ,arg / box  .code:n = \bool_set_true:N  \l_ducksay_eat_arg_box_bool
    ,arg / tab  .code:n =
      {
        \bool_set_false:N \l_ducksay_eat_arg_box_bool
        \bool_set_false:N \l_ducksay_eat_arg_tab_verb_bool
      }
    ,arg / tab* .code:n =
      {
        \bool_set_false:N \l_ducksay_eat_arg_box_bool
        \bool_set_true:N  \l_ducksay_eat_arg_tab_verb_bool
      }
    ,arg .initial:n = tab
    ,wd* .dim_set:N = \l_ducksay_msg_width_dim
    ,wd* .initial:n = -\c_max_dim
    ,wd* .value_required:n = true
    ,none          .bool_set:N = \l_ducksay_no_body_bool
    ,body-mirrored .bool_set:N = \l_ducksay_mirrored_body_bool
    ,ignore-body   .bool_set:N = \l_ducksay_ignored_body_bool
    ,body-x      .dim_set:N = \l_ducksay_body_x_offset_dim
    ,body-x      .value_required:n = true
    ,body-y      .dim_set:N = \l_ducksay_body_y_offset_dim
    ,body-y      .value_required:n = true
    ,body-to-msg .tl_set:N  = \l_ducksay_body_to_msg_align_body_tl
    ,msg-to-body .tl_set:N  = \l_ducksay_body_to_msg_align_msg_tl
    ,body-align .choice:
    ,body-align / l .meta:n = { body-to-msg = l , msg-to-body = l }
    ,body-align / c .meta:n = { body-to-msg = hc , msg-to-body = hc }
    ,body-align / r .meta:n = { body-to-msg = r , msg-to-body = r }
    ,body-align .initial:n = l
    ,msg-align   .choice:
    ,msg-align  / l .code:n = { \tl_set:Nn \l_ducksay_msg_align_tl { l } }
    ,msg-align  / c .code:n = { \tl_set:Nn \l_ducksay_msg_align_tl { c } }
    ,msg-align  / r .code:n = { \tl_set:Nn \l_ducksay_msg_align_tl { r } }
    ,msg-align  / j .code:n = { \tl_set:Nn \l_ducksay_msg_align_tl { j } }
    ,msg-align-l .tl_set:N  = \l_ducksay_msg_align_l_tl
    ,msg-align-l .initial:n = \raggedright
    ,msg-align-c .tl_set:N  = \l_ducksay_msg_align_c_tl
    ,msg-align-c .initial:n = \centering
    ,msg-align-r .tl_set:N  = \l_ducksay_msg_align_r_tl
    ,msg-align-r .initial:n = \raggedleft
    ,msg-align-j .tl_set:N  = \l_ducksay_msg_align_j_tl
    ,msg-align-j .initial:n = {}
    ,out-h   .tl_set:N  = \l_ducksay_output_h_pole_tl
    ,out-h   .initial:n = l
    ,out-v   .tl_set:N  = \l_ducksay_output_v_pole_tl
    ,out-v   .initial:n = vc
    ,out-x   .dim_set:N = \l_ducksay_output_x_offset_dim
    ,out-x   .value_required:n = true
    ,out-y   .dim_set:N = \l_ducksay_output_y_offset_dim
    ,out-y   .value_required:n = true
    ,t       .meta:n    = { out-v = t }
    ,c       .meta:n    = { out-v = vc }
    ,b       .meta:n    = { out-v = b }
    ,body*   .tl_set:N  = \l_ducksay_body_fount_tl
    ,msg*    .tl_set:N  = \l_ducksay_msg_fount_tl
    ,bubble* .tl_set:N  = \l_ducksay_bubble_fount_tl
    ,body*   .initial:n = \verbatim@font
    ,msg*    .initial:n = \verbatim@font
    ,bubble* .initial:n = \verbatim@font
    ,body    .code:n    = \tl_put_right:Nn \l_ducksay_body_fount_tl   { #1 }
    ,msg     .code:n    = \tl_put_right:Nn \l_ducksay_msg_fount_tl    { #1 }
    ,bubble  .code:n    = \tl_put_right:Nn \l_ducksay_bubble_fount_tl { #1 }
    ,MSG     .meta:n    = { msg  = #1 , bubble  = #1 }
    ,MSG*    .meta:n    = { msg* = #1 , bubble* = #1 }
    ,hpad    .int_set:N = \l_ducksay_hpad_int
    ,hpad    .initial:n = 2
    ,hpad    .value_required:n = true
    ,vpad    .int_set:N = \l_ducksay_vpad_int
    ,vpad    .value_required:n = true
    ,col     .tl_set:N  = \l_ducksay_msg_tabular_column_tl
    ,bubble-top-kern  .tl_set:N  = \l_ducksay_bubble_top_kern_tl
    ,bubble-top-kern  .initial:n = { -.5ex }
    ,bubble-top-kern  .value_required:n = true
    ,bubble-bot-kern  .tl_set:N  = \l_ducksay_bubble_bottom_kern_tl
    ,bubble-bot-kern  .initial:n = { .2ex }
    ,bubble-bot-kern  .value_required:n = true
    ,bubble-side-kern .tl_set:N  = \l_ducksay_bubble_side_kern_tl
    ,bubble-side-kern .initial:n = { 0.2em }
    ,bubble-side-kern .value_required:n = true
    ,bubble-delim-top     .tl_set:N  = \l_ducksay_bubble_delim_top_tl
    ,bubble-delim-left-1  .tl_set:N  = \l_ducksay_bubble_delim_left_a_tl
    ,bubble-delim-left-2  .tl_set:N  = \l_ducksay_bubble_delim_left_b_tl
    ,bubble-delim-left-3  .tl_set:N  = \l_ducksay_bubble_delim_left_c_tl
    ,bubble-delim-left-4  .tl_set:N  = \l_ducksay_bubble_delim_left_d_tl
    ,bubble-delim-right-1 .tl_set:N  = \l_ducksay_bubble_delim_right_a_tl
    ,bubble-delim-right-2 .tl_set:N  = \l_ducksay_bubble_delim_right_b_tl
    ,bubble-delim-right-3 .tl_set:N  = \l_ducksay_bubble_delim_right_c_tl
    ,bubble-delim-right-4 .tl_set:N  = \l_ducksay_bubble_delim_right_d_tl
    ,bubble-delim-top     .initial:n = { { - } }
    ,bubble-delim-left-1  .initial:n = (
    ,bubble-delim-left-2  .initial:n = /
    ,bubble-delim-left-3  .initial:n = |
    ,bubble-delim-left-4  .initial:n = \c_backslash_str
    ,bubble-delim-right-1 .initial:n = )
    ,bubble-delim-right-2 .initial:n = \c_backslash_str
    ,bubble-delim-right-3 .initial:n = |
    ,bubble-delim-right-4 .initial:n = /
  }
%    \end{macrocode}
%
%^^A<<<
%
% \subsubsection{Functions}^^A>>>
%
% \paragraph{Internal}^^A>>>
%
% \begin{macro}{\ducksay_evaluate_message_alignment_fixed_width_tabular:}^^A>>>
%    \begin{macrocode}
\cs_new:Npn \ducksay_evaluate_message_alignment_fixed_width_tabular:
  {
    \tl_if_empty:NT \l_ducksay_msg_tabular_column_tl
      {
        \tl_set:Nx \l_ducksay_msg_tabular_column_tl
          {
            >
            {
              \str_case:Vn \l_ducksay_msg_align_tl
                {
                  { l } { \exp_not:N \l_ducksay_msg_align_l_tl }
                  { c } { \exp_not:N \l_ducksay_msg_align_c_tl }
                  { r } { \exp_not:N \l_ducksay_msg_align_r_tl }
                  { j } { \exp_not:N \l_ducksay_msg_align_j_tl }
                }
              \exp_not:N \arraybackslash
            }
            p { \exp_not:N \l_ducksay_msg_width_dim }
          }
      }
  }
%    \end{macrocode}
% \end{macro}^^A<<<
%
% \begin{macro}{\ducksay_evaluate_message_alignment_fixed_width_vbox:}^^A>>>
%    \begin{macrocode}
\cs_new:Npn \ducksay_evaluate_message_alignment_fixed_width_vbox:
  {
    \tl_set:Nx \l_ducksay_msg_align_vbox_tl
      {
        \str_case:Vn \l_ducksay_msg_align_tl
          {
            { l } { \exp_not:N \l_ducksay_msg_align_l_tl }
            { c } { \exp_not:N \l_ducksay_msg_align_c_tl }
            { r } { \exp_not:N \l_ducksay_msg_align_r_tl }
            { j } { \exp_not:N \l_ducksay_msg_align_j_tl }
          }
      }
  }
%    \end{macrocode}
% \end{macro}^^A<<<
%
% \begin{macro}{\ducksay_calculate_msg_width_from_int:}^^A>>>
%    \begin{macrocode}
\cs_new:Npn \ducksay_calculate_msg_width_from_int:
  {
    \hbox_set:Nn \l_ducksay_tmpa_box { \l_ducksay_msg_fount_tl M }
    \dim_set:Nn \l_ducksay_msg_width_dim 
      { \l_ducksay_msg_width_int \box_wd:N \l_ducksay_tmpa_box }
  }
%    \end{macrocode}
% \end{macro}^^A<<<
%
% \begin{macro}{\ducksay_msg_tabular_begin:}^^A>>>
%    \begin{macrocode}
\cs_new:Npn \ducksay_msg_tabular_begin:
  {
    \ducksay_msg_tabular_begin_inner:V \l_ducksay_msg_tabular_column_tl
  }
\cs_new:Npn \ducksay_msg_tabular_begin_inner:n #1
  {
    \begin { tabular } { @{} #1 @{} }
  }
\cs_generate_variant:Nn \ducksay_msg_tabular_begin_inner:n { V }
%    \end{macrocode}
% \end{macro}^^A<<<
%
% \begin{macro}{\ducksay_msg_tabular_end:}^^A>>>
%    \begin{macrocode}
\cs_new:Npn \ducksay_msg_tabular_end:
  {
    \end { tabular }
  }
%    \end{macrocode}
% \end{macro}^^A<<<
%
% \begin{macro}{\ducksay_digest_options:n}^^A>>>
%    \begin{macrocode}
\cs_new:Npn \ducksay_digest_options:n #1
  {
    \keys_set:nn { ducksay } { #1 }
    \tl_if_empty:NT \l_ducksay_animal_tl
      { \keys_set:nn { ducksay } { default_animal } }
    \bool_if:NTF \l_ducksay_eat_arg_box_bool
      {
        \dim_compare:nNnTF { \l_ducksay_msg_width_dim } < { \c_zero_dim }
          {
            \int_compare:nNnTF { \l_ducksay_msg_width_int } < { \c_zero_int }
              {
                \cs_set_eq:NN
                  \ducksay_eat_argument:w \ducksay_eat_argument_hbox:w
              }
              {
                \cs_set_eq:NN
                  \ducksay_eat_argument:w \ducksay_eat_argument_vbox:w
                \ducksay_calculate_msg_width_from_int:
              }
          }
          {
            \cs_set_eq:NN \ducksay_eat_argument:w \ducksay_eat_argument_vbox:w
          }
      }
      {
        \dim_compare:nNnTF { \l_ducksay_msg_width_dim } < { \c_zero_dim }
          {
            \int_compare:nNnTF { \l_ducksay_msg_width_int } < { \c_zero_int }
              {
                \tl_if_empty:NT \l_ducksay_msg_tabular_column_tl
                  {
                    \str_case:Vn \l_ducksay_msg_align_tl
                      {
                        { l }
                          { \tl_set:Nn \l_ducksay_msg_tabular_column_tl { l } }
                        { c }
                          { \tl_set:Nn \l_ducksay_msg_tabular_column_tl { c } }
                        { r }
                          { \tl_set:Nn \l_ducksay_msg_tabular_column_tl { r } }
                        { j } {
                          \msg_error:nn { ducksay } { justify~unavailable }
                          \tl_set:Nn \l_ducksay_msg_tabular_column_tl { l }
                        }
                      }
                  }
              }
              {
                \ducksay_calculate_msg_width_from_int:
                \ducksay_evaluate_message_alignment_fixed_width_tabular:
              }
          }
          {
            \ducksay_evaluate_message_alignment_fixed_width_tabular:
          }
        \cs_set_eq:NN \ducksay_eat_argument:w \ducksay_eat_argument_tabular:w
      }
  }
%    \end{macrocode}
% \end{macro}^^A<<<
%
% \begin{macro}{\ducksay_set_bubble_top_kern:}^^A>>>
%    \begin{macrocode}
\cs_new:Npn \ducksay_set_bubble_top_kern:
  {
    \group_begin:
    \l_ducksay_bubble_fount_tl
    \exp_args:NNNx
    \group_end:
    \dim_set:Nn \l_ducksay_bubble_top_kern_dim
      { \dim_eval:n { \l_ducksay_bubble_top_kern_tl } }
  }
%    \end{macrocode}
% \end{macro}^^A<<<
%
% \begin{macro}{\ducksay_set_bubble_bottom_kern:}^^A>>>
%    \begin{macrocode}
\cs_new:Npn \ducksay_set_bubble_bottom_kern:
  {
    \group_begin:
    \l_ducksay_bubble_fount_tl
    \exp_args:NNNx
    \group_end:
    \dim_set:Nn \l_ducksay_bubble_bottom_kern_dim
      { \dim_eval:n { \l_ducksay_bubble_bottom_kern_tl } }
  }
%    \end{macrocode}
% \end{macro}^^A<<<
%
% \begin{macro}{\ducksay_shipout:}^^A>>>
%    \begin{macrocode}
\cs_new_protected:Npn \ducksay_shipout:
  {
    \hbox_set:Nn \l_ducksay_tmpa_box
      { \l_ducksay_bubble_fount_tl \l_ducksay_bubble_delim_top_tl }
    \int_set:Nn \l_ducksay_msg_width_int
      {
        \fp_eval:n
          {
            ceil 
              ( \box_wd:N \l_ducksay_msg_box / \box_wd:N \l_ducksay_tmpa_box )
          }
      }
    \group_begin:
    \l_ducksay_bubble_fount_tl
    \exp_args:NNNx
    \group_end:
    \int_set:Nn \l_ducksay_msg_height_int
      {
        \int_max:nn
          {
            \fp_eval:n
              {
                ceil
                  (
                    (
                      \box_ht:N \l_ducksay_msg_box
                      + \box_dp:N \l_ducksay_msg_box
                    )
                    / ( \arraystretch * \baselineskip )
                  )
              }
            + \l_ducksay_vpad_int
          }
          { \l_ducksay_msg_height_int }
      }
    \hcoffin_set:Nn \l_ducksay_bubble_open_coffin
      {
        \l_ducksay_bubble_fount_tl
        \begin{tabular}{@{}l@{}}
          \int_compare:nNnTF { \l_ducksay_msg_height_int } = { \c_one_int }
            {
              \l_ducksay_bubble_delim_left_a_tl
            }
            {
              \l_ducksay_bubble_delim_left_b_tl\\
              \int_step_inline:nnn
                { 3 } { \l_ducksay_msg_height_int }
                {
                  \kern-\l_ducksay_bubble_side_kern_tl
                  \l_ducksay_bubble_delim_left_c_tl
                  \\
                }
              \l_ducksay_bubble_delim_left_d_tl
            }
        \end{tabular}
      }
    \hcoffin_set:Nn \l_ducksay_bubble_close_coffin
      {
        \l_ducksay_bubble_fount_tl
        \begin{tabular}{@{}r@{}}
          \int_compare:nNnTF { \l_ducksay_msg_height_int } = { \c_one_int }
            {
              \l_ducksay_bubble_delim_right_a_tl
            }
            {
              \l_ducksay_bubble_delim_right_b_tl \\
              \int_step_inline:nnn
                { 3 } { \l_ducksay_msg_height_int }
                {
                  \l_ducksay_bubble_delim_right_c_tl
                  \kern-\l_ducksay_bubble_side_kern_tl
                  \\
                }
              \l_ducksay_bubble_delim_right_d_tl
            }
        \end{tabular}
      }
    \hcoffin_set:Nn \l_ducksay_bubble_top_coffin
      {
        \l_ducksay_bubble_fount_tl
        \int_step_inline:nn { \l_ducksay_msg_width_int + \l_ducksay_hpad_int }
          { \l_ducksay_bubble_delim_top_tl }
      }
    \hcoffin_set:Nn \l_ducksay_msg_coffin { \box_use:N \l_ducksay_msg_box }
    \bool_if:NF \l_ducksay_no_body_bool
      {
        \hcoffin_set:Nn \l_ducksay_body_coffin
          {
            \frenchspacing
            \l_ducksay_body_fount_tl
            \begin{tabular} { @{} l @{} }
              \l_ducksay_animal_tl
            \end{tabular}
          }
        \bool_if:NT \l_ducksay_mirrored_body_bool
          {
            \coffin_scale:Nnn \l_ducksay_body_coffin
              { -\c_one_int } { \c_one_int }
            \str_case:Vn \l_ducksay_body_to_msg_align_body_tl
              {
                { l } { \tl_set:Nn \l_ducksay_body_to_msg_align_body_tl { r } }
                { r } { \tl_set:Nn \l_ducksay_body_to_msg_align_body_tl { l } }
              }
          }
      }
    \dim_set:Nn \l_ducksay_hpad_dim
      {
        (
          \coffin_wd:N \l_ducksay_bubble_top_coffin
          - \coffin_wd:N \l_ducksay_msg_coffin
        ) / 2
      }
    \coffin_join:NnnNnnnn
      \l_ducksay_msg_coffin         { l } { vc }
      \l_ducksay_bubble_open_coffin { r } { vc }
      { - \l_ducksay_hpad_dim } { \c_zero_dim }
    \coffin_join:NnnNnnnn
      \l_ducksay_msg_coffin          { r } { vc }
      \l_ducksay_bubble_close_coffin { l } { vc }
      { \l_ducksay_hpad_dim } { \c_zero_dim }
    \ducksay_set_bubble_top_kern:
    \ducksay_set_bubble_bottom_kern:
    \coffin_join:NnnNnnnn
      \l_ducksay_msg_coffin        { hc } { t }
      \l_ducksay_bubble_top_coffin { hc } { b }
      { \c_zero_dim } { \l_ducksay_bubble_top_kern_dim }
    \coffin_join:NnnNnnnn
      \l_ducksay_msg_coffin        { hc } { b }
      \l_ducksay_bubble_top_coffin { hc } { t }
      { \c_zero_dim } { \l_ducksay_bubble_bottom_kern_dim }
    \bool_if:NF \l_ducksay_no_body_bool
      {
        \bool_if:NTF \l_ducksay_ignored_body_bool
          { \coffin_attach:NVnNVnnn }
          { \coffin_join:NVnNVnnn   }
          \l_ducksay_msg_coffin  \l_ducksay_body_to_msg_align_msg_tl  { b }
          \l_ducksay_body_coffin \l_ducksay_body_to_msg_align_body_tl { t }
          { \l_ducksay_body_x_offset_dim } { \l_ducksay_body_y_offset_dim }
      }
    \coffin_typeset:NVVnn \l_ducksay_msg_coffin
      \l_ducksay_output_h_pole_tl \l_ducksay_output_v_pole_tl
      { \l_ducksay_output_x_offset_dim } { \l_ducksay_output_y_offset_dim }
    \group_end:
  }
%    \end{macrocode}
% \end{macro}^^A<<<
%
% ^^A\subparagraph{Questionable Syntax Introducing Functions (Hacks)}^^A>>>
%
%^^A<<<
%
% \subparagraph{Message Reading Functions}^^A>>>
%
% Version 2 has different ways of reading the message argument of \cs{ducksay}
% and \cs{duckthink}. They all should allow almost arbitrary content and the
% height and width are set based on the dimensions.
%
% \begin{macro}{\ducksay_eat_argument_tabular:w}^^A>>>
%    \begin{macrocode}
\cs_new:Npn \ducksay_eat_argument_tabular:w
  {
    \bool_if:NTF \l_ducksay_eat_arg_tab_verb_bool
      { \ducksay_eat_argument_tabular_verb:w }
      { \ducksay_eat_argument_tabular_normal:w }
  }
%    \end{macrocode}
% \end{macro}^^A<<<
%
% \begin{macro}{\ducksay_eat_argument_tabular_inner:w}^^A>>>
%    \begin{macrocode}
\cs_new:Npn \ducksay_eat_argument_tabular_inner:w #1
  {
    \hbox_set:Nn \l_ducksay_msg_box
      {
        \l_ducksay_msg_fount_tl
        \ducksay_msg_tabular_begin:
          #1
        \ducksay_msg_tabular_end:
      }
    \ducksay_shipout:
  }
%    \end{macrocode}
% \end{macro}^^A<<<
%
% \begin{macro}{\ducksay_eat_argument_tabular_verb:w}^^A>>>
%    \begin{macrocode}
\NewDocumentCommand \ducksay_eat_argument_tabular_verb:w
  { >{ \ducksay_process_verb_newline:nnn { ~ } { ~ \par } } +v }
  { \ducksay_eat_argument_tabular_inner:w { \scantokens { #1 } } }
%    \end{macrocode}
% \end{macro}^^A<<<
%
% \begin{macro}{\ducksay_eat_argument_tabular_normal:w}^^A>>>
%    \begin{macrocode}
\NewDocumentCommand \ducksay_eat_argument_tabular_normal:w { +m }
  { \ducksay_eat_argument_tabular_inner:w { #1 } }
%    \end{macrocode}
% \end{macro}^^A<<<
%
% \begin{macro}{\ducksay_eat_argument_hbox:w}^^A>>>
%    \begin{macrocode}
\cs_new_protected_nopar:Npn \ducksay_eat_argument_hbox:w
  {
    \group_begin:
    \afterassignment \ducksay_eat_argument_hbox_inner:w
    \let \l_ducksay_nothing =
  }
%    \end{macrocode}
% \end{macro}^^A<<<
%
% \begin{macro}{\ducksay_eat_argument_hbox_inner:w}^^A>>>
%    \begin{macrocode}
\cs_new_protected_nopar:Npn \ducksay_eat_argument_hbox_inner:w
  {
    \group_end:
    \setbox \l_ducksay_msg_box \hbox \c_group_begin_token
      \group_insert_after:N \ducksay_shipout:
      \l_ducksay_msg_fount_tl
  }
%    \end{macrocode}
% \end{macro}^^A<<<
%
% \begin{macro}{\ducksay_eat_argument_vbox:w}^^A>>>
%    \begin{macrocode}
\cs_new_protected_nopar:Npn \ducksay_eat_argument_vbox:w
  {
    \ducksay_evaluate_message_alignment_fixed_width_vbox:
    \group_begin:
    \afterassignment \ducksay_eat_argument_vbox_inner:w
    \let \l_ducksay_nothing =
  }
%    \end{macrocode}
% \end{macro}^^A<<<
%
% \begin{macro}{\ducksay_eat_argument_vbox_inner:w}^^A>>>
%    \begin{macrocode}
\cs_new_protected_nopar:Npn \ducksay_eat_argument_vbox_inner:w
  {
    \group_end:
    \setbox \l_ducksay_msg_box \vbox \c_group_begin_token
      \hsize \l_ducksay_msg_width_dim
      \linewidth \hsize
      \group_insert_after:N \ducksay_shipout:
      \l_ducksay_msg_fount_tl
      \l_ducksay_msg_align_vbox_tl
      \@afterindentfalse
      \@afterheading
  }
%    \end{macrocode}
% \end{macro}^^A<<<
%
%^^A<<<
%
% \subparagraph{Generating Variants of External Functions}^^A>>>
%
%    \begin{macrocode}
\cs_generate_variant:Nn \coffin_join:NnnNnnnn { NVnNVnnn }
\cs_generate_variant:Nn \coffin_attach:NnnNnnnn { NVnNVnnn }
\cs_generate_variant:Nn \coffin_typeset:Nnnnn { NVVnn }
\cs_generate_variant:Nn \tl_if_eq:nnT { VnT }
\cs_generate_variant:Nn \str_case:nn { Vn }
\cs_generate_variant:Nn \regex_replace_all:NnN { Nnc }
%    \end{macrocode}
%
%^^A<<<
%
%^^A<<<
%
% \paragraph{Document level}^^A>>>
%
% \begin{macro}{\ducksay}^^A>>>
%    \begin{macrocode}
\NewDocumentCommand \ducksay { O{} }
  {
    \group_begin:
      \tl_set:Nn \l_ducksay_say_or_think_tl { say }
      \ducksay_digest_options:n { #1 }
      \ducksay_eat_argument:w
  }
%    \end{macrocode}
% \end{macro}^^A<<<
%
% \begin{macro}{\duckthink}^^A>>>
%    \begin{macrocode}
\NewDocumentCommand \duckthink { O{} }
  {
    \group_begin:
      \tl_set:Nn \l_ducksay_say_or_think_tl { think }
      \ducksay_digest_options:n { #1 }
      \ducksay_eat_argument:w
  }
%    \end{macrocode}
% \end{macro}^^A<<<
%
%^^A<<<
%
%^^A<<<
%
%    \begin{macrocode}
%</code.v2>
%    \end{macrocode}^^A<<<
%
% \SetVersion{}%
% \subsection{Definition of the Animals}^^A>>>
%
%    \begin{macrocode}
%<*animals>
%^^A some of the below are from http://ascii.co.uk/art/kangaroo
\AddAnimal{duck}%>>>
{  \
    \   __
      >(' )
        )/
       /(
      /  `----/
      \  ~=- /
    ~^~^~^~^~^~^~^}%<<<
\AddAnimal{small-duck}%>>>
{  \
    \
      >()_
       (__)__ _}%<<<
\AddAnimal{duck-family}%>>>
{  \
    \   __
      >(' )
        )/
       /(
      /  `----/  -()_  >()_
    __\__~=-_/__ _(__)__(__)__ _}%<<<
\AddAnimal{cow}%>>>
{  \  ^__^
    \ (oo)\_______
      (__)\       )\/\
          ||----w |
          ||     ||}%<<<
\AddAnimal{head-in}%>>>
{  \  
    \ ^__^         /
      (oo)\_______/  ________
      (__)\       )=(  ___|_ \____
          ||----w |  \ \    \____ |
          ||     ||   ||         ||}%<<<
\AddAnimal{sodomized}%>>>
{  \             _
    \           (_)
      ^__^       / \
      (oo)\_____/_\ \
      (__)\       ) /
          ||----w ((
          ||     ||>>}%<<<
\AddAnimal{tux}%>>>
{  \
    \  .--. 
      |o_o |
      |\_/ |
     //   \ \
    (|     | )
   /'\_   _/`\
   \___)=(___/}%<<<
\AddAnimal{pig}%>>>
+  \     _//| .-~~~-.
    \ _/oo  }        }-@
     ('')_  }        |
      `--'| { }--{  }
           //_/  /_/+%<<<
\AddAnimal{frog}%>>>
{   \
     \ (.)_(.)
    _ (   _   ) _
   / \/`-----'\/ \
 __\ ( (     ) ) /__
 )   /\ \._./ /\   (
  )_/ /|\   /|\ \_(}%<<<
\AddAnimal{snowman}%>>>
{  \
    \_[_]_
      (")
   >-( : )-<
    (__:__)}%<<<
\AddAnimal{hedgehog}%>>>
{  \    .\|//||\||.
    \  |/\/||/|//|/|
      /. `|/\\|/||/||
     o__,_|//|/||\||'}%<<<
\AddAnimal{kangaroo}%>>>
{  \
    \ _,'   ___
     <__\__/   \
        \_  /  _\
          \,\ / \\
            //   \\
          ,/'     `\_,}%<<<
%^^A http://chris.com/ascii/index.php?art=animals/rabbits
\AddAnimal{rabbit}%>>>
{ \     / \`\         __
   \   |  \ `\      /`/ \
    \  \_/`\  \-"-/` /\  \
            |       |  \  |
            (d     b)   \_/
            /       \
        ,".|.'.\_/.'.|.",
       /   /\' _|_ '/\   \
       |  /  '-`"`-'  \  |
       | |             | |
       | \    \   /    / |
        \ \    \ /    / /
         `"`\   :   /'"`
             `""`""`}%<<<
\AddAnimal{bunny}%>>>
{ \
   \      /
      /\ /
       ( )
     .( o ).}%<<<
\AddAnimal{small-rabbit}%>>>
{  \
    \ _//
     (')---.
      _/-_( )o}%<<<
\AddAnimal{dragon}%>>>
{     \                    / \  //\
       \    |\___/|      /   \//  \\
        \   /0  0  \__  /    //  | \ \    
           /     /  \/_/    //   |  \  \  
           @_^_@'/   \/_   //    |   \   \ 
           //_^_/     \/_ //     |    \    \
        ( //) |        \///      |     \     \
      ( / /) _|_ /   )  //       |      \     _\
    ( // /) '/,_ _ _/  ( ; -.    |    _ _\.-~        .-~~~^-.
  (( / / )) ,-{        _      `-.|.-~-.           .~         `.
 (( // / ))  '/\      /                 ~-. _ .-~      .-~^-.  \
 (( /// ))      `.   {            }                   /      \  \
  (( / ))     .----~-.\        \-'                 .~         \  `. \^-.
             ///.----..>        \             _ -~             `.  ^-`  ^-_
               ///-._ _ _ _ _ _ _}^ - - - - ~                     ~-- ,.-~
                                                                  /.-~}%<<<
%^^A http://www.ascii-art.de/ascii/def/dogs.txt
\AddAnimal{dog}%>>>
{  \     __
    \ .-'\/\
       "\   '------.
     ___/       (  .'_____
    '-----'"""'------"""""'}%<<<
%^^A http://ascii.co.uk/art/squirrel
\AddAnimal{squirrel}%>>>
{  \           ,;:;;,
    \    ,    ;;;;;
      .=',    ;:;;:,
     /_', "=. ';:;:;
     @=:__,  \,;:;:'
       _(\.=  ;:;;'
      `"_(  _/="`
       `"'``}%<<<
\AddAnimal{snail}%>>>
{  \
    \          .-""-.
      oo      ; .-.  :
       \\__..-: '.__.')._
        "-._.._'.__.-'_.."}%<<<
%^^A http://www.ascii-art.de/ascii/uvw/unicorn.txt
\AddAnimal{unicorn}%>>>
{   \
     \       /((((((\\\\
     ---====((((((((((\\\\\
          ((           \\\\\\\
          ( (*    _/      \\\\\\\
            \    /  \      \\\\\\_         __,,__
             |  |   |       </    "------""     ((\\\\
             o_|   /        /                      \ \\\\    \\\\\\\
                  |  ._    (                        \ \\\\\\\\\\\\\\\\
                  | /                       /       /    \\\\\\\     \\
          .______/\/     /                 /       /         \\\
         / __.____/    _/          ___----(       /\
        / / / ________/:______,---'        \     /  \_
       / /  \ \                             \   \ \_  \
      ( <    \ \                             >  /    \ \
       \/      \\_                          / /       > )
                \_|                        / /       / /
                                         _//       _//
                                       /_|       /_|}%<<<
%^^A https://asciiart.website//index.php?art=animals/other%20(water)
\AddAnimal{whale}%>>>
{  \                |-.
    \    .-""-._     \ \.--|
     \  /       `-..__)  ,-'
       |     .          /
        \--.__,   .__.,'
         `-.___'._\_.'}%<<<
%^^A from http://www.ascii-art.de/ascii/s/starwars.txt :
\AddAnimal{yoda}%>>>
{   \
     \             ____
      \         _.' :  `._
            .-.'`.  ;   .'`.-.
   __      / : ___\ ;  /___ ; \      __
 ,'_ ""--.:__;".-.";: :".-.":__;.--"" _`,
 :' `.t""--.. '<@.`;_  ',@>` ..--""j.' `;
      `:-.._J '-.-'L__ `-- ' L_..-;'
        "-.__ ;  .-"  "-.  : __.-"
            L ' /.------.\ ' J
             "-.   "--"   .-"
            __.l"-:_JL_;-";.__
         .-j/'.;  ;""""  / .'\"-.
       .' /:`. :  :     /.".'';  `.
    .-"  / ;`.".  :    ."."   :    "-.
 .+"-.  : :   ".".". ."."      ;-._   \
 ; \  `.; ; .   "."-"."        : : "+. ;
 :  ;   ; ;  .   ."."    ;     : ;  : \:
 ;  :   ; :     / /     /  ,   ;:   ;  :
: \  ;  :  ;   ; /     :  ,   : ;  /  ::
;  ; :   ; :  ; ;      ;      ;   :   ;:
:  :  ;  :  ;. ;      '      : :  ;  : ;
;\    :   ; : .          ,   ; ;     ; ;
: `."-;   :  ;      .   ;   :  ;    /  ;
 ;    -:   ; :      ,  ,    ;  : .-"   :
 :\     \  :  ;    ,       : \.-"      :
  ;`.    \  ; :   .   ,    ;.'_..--  / ;
  :  "-.  "-:  ;     ,    :/."      .'  :
   \         \ :    :     ;/  __        :
    \       .-`.\        /t-""  ":-+.   :
     `.  .-"    `l    __/ /`. :  ; ; \  ;
       \   .-" .-"-.-"  .' .'j \  /   ;/
        \ / .-"   /.     .'.' ;_:'    ;
         :-""-.`./-.'     /    `.___.'
               \ `t  ._  /
                "-.t-._:'}%<<<
\AddAnimal{yoda-head}%>>>
{   \
     \             ____
      \         _.' :  `._
            .-.'`.  ;   .'`.-.
   __      / : ___\ ;  /___ ; \      __
 ,'_ ""--.:__;".-.";: :".-.":__;.--"" _`,
 :' `.t""--.. '<@.`;_  ',@>` ..--""j.' `;
      `:-.._J '-.-'L__ `-- ' L_..-;'
        "-.__ ;  .-"  "-.  : __.-"
            L ' /.------.\ ' J
             "-.   "--"   .-"
            __.l"-:_JL_;-";.__
         .-j/'.;  ;""""  / .'\"-.
       .' /:`. :  :     /.".'';  `.
    .-"  / ;`.".  :    ."."   :    "-.
 .+"-.  : :   ".".". ."."      ;-._   \}%<<<
%^^A from https://www.ascii-code.com/ascii-art/movies/star-wars.php
\AddAnimal{small-yoda}%>>>
{  \
    \
    __.-._
    '-._"7'
     /'.-c
     |  /T
    _)_/LI}%<<<
\AddAnimal{r2d2}%>>>
{  \
    \ ,-----.
    ,'_/_|_\_`.
   /<<::8[O]::>\
  _|-----------|_
 |  | ====-=- |  |
 |  | -=-==== |  |
 \  | ::::|()||  /
  | | ....|()|| |
  | |_________| |
  | |\_______/| |
 /   \ /   \ /   \
 `---' `---' `---'}%<<<
\AddAnimal{vader}%>>>
{  \     _.-'~~~~~~`-._
    \   /      ||      \
       /       ||       \
      |        ||        |
      | _______||_______ |
      |/ ----- \/ ----- \|
     /  (     )  (     )  \
    / \  ----- () -----  / \
   /   \      /||\      /   \
  /     \    /||||\    /     \
 /       \  /||||||\  /       \
/_        \O========O/        _\
  `--...__|`-._  _.-'|__...--'
          |    `'    |}%<<<
%</animals>
%    \end{macrocode}^^A<<<
%
%^^A<<<
%
% \end{implementation}^^A<<<
%
% \closingpage
%
\endinput
%
^^A vim: ft=tex fdm=marker fmr=>>>,<<<
